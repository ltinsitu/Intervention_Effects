\documentclass[11pt,final,hyperref={pdfpagelabels=false}]{beamer}
\usepackage{pgfpages}
\pgfpagesuselayout{4 on 1}[a4paper,landscape,border shrink=2mm]
\usepackage[english]{babel}
\usepackage[utf8x]{inputenc}
\usepackage{amsmath,amsthm,amssymb,latexsym}
\usepackage{graphicx}
\usepackage{csquotes}
\usepackage{natbib}
\usepackage{multicol}
\usepackage{stmaryrd}
\usepackage{enumitem}


%%%%%%%%%%%%%%%%%%%%%%%%%%%%%%%%%%%%%%%%%%%%%%%%%%%%%%%%%%%%%%%%%%%%%%%%%%%%%%%%%%%%%
% Pour éviter le bug 'rmfamily invalid in math mode' apparemment causé par gb4e
% \makeatletter
% \def\***@fontshape{}
% \makeatother
%%%%%%%%%%%%%%%%%%%%%%%%%%%%%%%%%%%%%%%%%%%%%%%%%%%%%%%%%%%%%%%%%%%%%%%%%%%%%%%%%%%%%

\usepackage{gb4e}%Fuckin' gb4e always at the end


%Newcommands
%%%%%%%%%%%%%%%%%%%%%%%%%%%%%%%%%%%%%%%%%%%%%%%%%%%%%%%%%%%%%%%%%%%%%%%%%%%%%%%%%%%%%
\newcommand{\reff}[1]{(\ref{#1})}
\newcommand{\valsem}[2][]{$\llbracket$#2$\rrbracket^{#1}$}
\newcommand{\tuple}[1]{\ensuremath{ \left \langle #1 \right \rangle }}
\newcommand{\exs}[2][1]{\begin{exe}\ex\label{#1} \begin{xlist}#2\end{xlist}\end{exe}}
\newcommand{\types}[1]{\ensuremath{ \left \langle \texttt{#1} \right \rangle }}
\newcommand{\rmsc}[1]{\text{\textsc{#1}}}
\newcommand{\rmbf}[1]{\text{\textbf{#1}}}
%%%%%%%%%%%%%%%%%%%%%%%%%%%%%%%%%%%%%%%%%%%%%%%%%%%%%%%%%%%%%%%%%%%%%%%%%%%%%%%%%%%%%


%Pour avoir des frametitles du nom de la section et numérotées
%%%%%%%%%%%%%%%%%%%%%%%%%%%%%%%%%%%%%%%%%%%%%%%%%%%%%%%%%%%%%%%%%%%%%%%%%%%%%%%%%%%%%
\newcounter{framesecnum}

\newcommand\framesecnum{%
    \frametitle{\refstepcounter{framesecnum}~\insertsection~({\Roman{framesecnum}})}
}
\resetcounteronoverlays{framesecnum}

\AtBeginSection{\setcounter{framesecnum}{1}}

\newcounter{framesubsecnum}

\newcommand\framesubsecnum{%
    \frametitle{\refstepcounter{framesubsecnum}~\insertsubsection~({\Roman{framesubsecnum}})}
}
\resetcounteronoverlays{framesubsecnum}

\AtBeginSubsection{\setcounter{framesubsecnum}{1}}

%%%%%%%%%%%%%%%%%%%%%%%%%%%%%%%%%%%%%%%%%%%%%%%%%%%%%%%%%%%%%%%%%%%%%%%%%%%%%%%%%%%%%


%Theme related
%%%%%%%%%%%%%%%%%%%%%%%%%%%%%%%%%%%%%%%%%%%%%%%%%%%%%%%%%%%%%%%%%%%%%%%%%%%%%%%%%%%%%
% \setbeamertemplate{blocks}[rounded][shadow=true]
\setbeamertemplate{navigation symbols}{\usebeamerfont{footline}%
    \usebeamercolor[fg]{black}%
    \hspace{1em}%
    \insertframenumber/\inserttotalframenumber}

% \usefonttheme{default}
\resetcounteronoverlays{exx}

\setbeamercolor{blocktitle}{bg=red,fg=red}
\useinnertheme[shadow=true]{rounded}
\usecolortheme{whale}
\useoutertheme[height=20pt]{sidebar}
\setbeamercolor{frametitle}{fg=white}

%Itemize labels (enumitem package)
%%%%%%%%%%%%%%%%%%%%%%%%%%%%%%%%%%%%%%%%%%%%%%%%%%%%%%%%%%%%%%%%%%%%%%%%%%%%%%%%%%%%%
\setlist[itemize]{label={$\bullet$}}
\setlist[itemize,2]{label={--}}
\setlist[itemize,3]{label={$\star$}}
%%%%%%%%%%%%%%%%%%%%%%%%%%%%%%%%%%%%%%%%%%%%%%%%%%%%%%%%%%%%%%%%%%%%%%%%%%%%%%%%%%%%%


 
\title[S. Beck - Intervention Effects]{\large Sigrid \raisebox{-.45\height}{\includegraphics[scale=0.027]{becks.png}} paper:\\\emph{Intervention Effects Follow from Focus Interpretation} (2006)}
\author{Karoliina Lohiniva, Lucas Tual\\University of Geneva}
\date{December 10, 2015}


\begin{document}
\begin{frame}
\maketitle
\end{frame}

\section{Introduction}

\begin{frame}
    \frametitle{Outline}
    \tableofcontents
\end{frame}


\begin{frame}{Introduction}
\begin{itemize}
\item Intervention effects in wh-questions are due to semantic uninterpretability
\item This semantic uninterpretability is due to the semantics of wh-expressions (their lack of ordinary semantic value) 
    \begin{itemize}
    \item In intervention configurations, Q is separated from the wh-expression by an intervening focus OP and cannot evaluate it
    \item The resetting of the focus semantic value to the ordinary semantic value by the focus OP will lead to an undefined semantic value
    \item The undefinedness percolates up through the whole structure 
    \end{itemize}
\item (Semantic) focus interpretation is argued to be the correlate of (syntactic) feature movement (Pesetsky 2000)
\end{itemize}
\end{frame}


\begin{frame}\framesecnum
\begin{itemize}
\item In general: a wh-phrase in situ may not be c-commanded by a focusing or quantificational element
	\begin{itemize}
	\item More specifically: the binder of a wh-phrase in situ (Q) must not be outside the scope of a focus-sensitive operator:
	    \begin{itemize}
	    \item OP $>$ Q $>$ wh
	    \item *Q $>$ OP $>$ wh
	    \end{itemize}
	\end{itemize}
\item In general: tension between universality and variation
	\begin{itemize}
	\item Universality of the phenomenon
	\item Parameters of variation
		\begin{itemize}
		\item Syntactic configurations where intervention effects arise
		\item The set of problematic intervenors
		\item The set of wh-expressions sensitive to intervention
		\end{itemize}
	\end{itemize}
\end{itemize}
\end{frame}


\subsection{Some examples}
\begin{frame}{Some examples}

\begin{itemize}
\item In English, intervention effects only appear in multiple wh-questions that avoid superiority effects
\end{itemize}

	\begin{exe}
  	 \ex \cite{Pes00}
   		\begin{xlist}
		\ex {\footnotesize{Which girl did only Mary introduce which girl to \_ ?}}
		\ex[??]{\footnotesize{Which boy did only Mary introduce which girl to \_ ?}}
   		\end{xlist}
   	\end{exe}
   	
\begin{itemize}
\item In English, but not in Thai, negation is an intervener
\end{itemize}
    \begin{exe}
	\ex \cite{Pes00} for English, \cite{Ruang02} for Thai 
		\begin{xlist}
		\ex[??]{\footnotesize{Which diplomat should I not discuss which issue with \_ ?}}
		\ex {\footnotesize{\gll Nit may sii ?aray\\
		Nit not buy what \\
		`What didn't Nit buy?'}}
		\end{xlist}
		\end{exe}
   
\end{frame}

%%%%%%%%%%%%%%%%%%%%%%%%%%%%%%
\begin{frame}{Some examples (II)}
\begin{itemize}
\item In Mandarin, who/what are sensitive to intervention, while which-phrases are not
\end{itemize}
	\begin{exe}
 
	\ex Beck for Mandarin, \cite{Soh01}
		\begin{xlist}
		\ex[\%]{\gll zhiyou Lili kan-le shenme? \\
		only Lili read-Asp what\\
		}
		\ex[?*]{\gll zhiyou Lili kan-le na-ben shu? \\
		only Lili read-Asp which-CL book\\
		}
		\end{xlist}
	\end{exe}
   
\end{frame}

\begin{frame}{Universal interveners}
Beck cites \cite{Kim02}, who claims that there is a core set of interveners, that are crosslinguistically stable. These are \emph{only}, \emph{even} and \emph{also}, and they should always give rise to intervention effects in every language.
\end{frame}


%%Karo


%%Lucas
\section{The Theory of Focus Interpretation}
\begin{frame}{The Theory of Focus Interpretation}
A theory for the interpretation of focus was developed by \cite{Roo85,Roo92}. Beck uses a variant of Rooth's account, that was developed by \cite{Kra91} and \cite{Wol96}.

According to Rooth, a constituent marked by focus will generate a set of alternatives to that constituent.

\begin{exe}
  \ex\label{focexample} John introduced Sue_F to Bill.
\end{exe}

The set of alternative propositions for the sentence in \reff{focexample} will be the following:

\begin{exe}
  \ex {\small{$\{\lambda{w_s}.~[\text{John introduced $x$ to Bill in $w$}] : x \in D_e\}$}}
\end{exe}

\emph{Only} is a focus sensitive operator that uses the set of alternatives.
\end{frame}

\begin{frame}\framesecnum
\begin{exe}
  \ex \enquote{\emph{only} $\phi$} is true only if given any true proposition in the set of alternatives to $\phi$, that proposition equals the proposition expressed by $\phi$.
\end{exe}

Thus, the sentence containing \emph{only} in \reff{sentonly} is true only if for any true proposition $\lambda{w_s}.~[\text{John introduced $x$ to Bill in $w$}]$\\ in the set of alternatives generated by the focus constituent, that proposition equals $\lambda{w^\prime_s}.~[\text{John introduced Sue to Bill in $w^\prime$}]$

\begin{exe}
  \ex\label{sentonly} John only introduced Sue_F to Bill.
\end{exe}

\begin{block}{Question}
How to derive the set of alternatives compositionally?  
\end{block}

  
\end{frame}

\begin{frame}\framesecnum
\cite{Kra91}'s version of Rooth's theory makes use of assignment variables. $g$ is the ordinary variable assignment. The semantic value of a logical form $\alpha$ is relativized to an assignment function $g$: \valsem[g]{$\alpha$}. One of the roles of $g$ is to assign a value to the index on a pronoun.

\begin{exe}
  \ex \valsem[g]{he_j} $=$ $g(j)$\\[0.2cm]
      \valsem[g]{He loves Bill} $=$ $\lambda{w_s}.~[g(j) \text{ loves Bill in }w]$
\end{exe}

Focus Features $F$ also bear an index. Another assignment variable, distinguished from $g$, is needed. The task of the variable assignment $h$, that may only be applied to indices on foci, is to build the set of focus alternatives.
\end{frame}

\begin{frame}\framesecnum{}
Relativization of the semantic value of a focused element $\alpha_{F_{i}}$ to a focus variable assignment $h$ produces an alternative $h(i)$.

If $\alpha_{F_{i}}$ is not relativized to an alternative assignment, $F_i$ is semantically inert.
\begin{exe}
  \ex {\small{Semantics of the Focus Feature $F_i$:}}
            \begin{xlist}
                \ex {\small{\valsem[g,h]{$\alpha_{F_{i}}$} defined iff $i \notin$ Dom($g$) \& $i \in$ Dom($h$)\\
                    \valsem[g,h]{$\alpha_{F_{i}}$} $=$ $h(i)$}}
                \ex {\small{\valsem[g]{$\alpha_{F_{i}}$} defined iff $i \notin$ Dom($g$)\\
                    \valsem[g]{$\alpha_{F_{i}}$} $=$ \valsem[g]{$\alpha$}}}
            \end{xlist}
  \ex The set of alternatives to $\phi$ is:\\
      $\{\text{\valsem[g,h]{$\phi$}} : h \in H\}$\\
      where $H$ is the set of focus variable assignments.
\end{exe}
\end{frame}

\begin{frame}\framesecnum{}
According to Beck, the contribution of focus can be evaluated by two focus-sensitive operators: either the squiggle $\sim$, or the question operator $Q$.

\medskip

The squiggle operator \citep{Roo92}: $\sim$

This operator defines at which syntactic level the focus should be interpreted, and has two semantic contributions:
  \begin{itemize}
    \item to semantically evaluate all foci in its scope unselectively (see \reff{semevaluation} on the next slide),
    \item and to neutralize the contribution of these foci by resetting the focus semantic value of the sister of $\sim$ to its ordinary semantic value (see \reff{semreset} on the next slide).
  \end{itemize} 
\end{frame}

\begin{frame}\framesecnum{}
Consider the semantics behind the sentence in \reff{onlyexample}, whose LF is \reff{onlyexampleLF}. The variable $C$ is a focus anaphor, that will be used both by \enquote{\emph{only}} and the squiggle operator.
\begin{exe}
  \ex 
    \begin{xlist}
      \ex\label{onlyexample} Only JOHN left.
      \ex\label{onlyexampleLF} \lb{} only $C$ \lb{} \lb{} $\sim$ $C$ \rb{} \lb{} John$_{F_{i}}$ left \rb{} \rb{} \rb{}
    \end{xlist}
  \ex If $X = \lb{} \lb{} \hspace{-1.2ex} \sim C\rb{} Y \rb{}$, then
    \begin{xlist}
      \ex\label{semevaluation} \valsem[g]{$X$} $=$ \valsem[g]{$Y$} if $g(C) \subseteq \{$\valsem[g,h^\prime]{Y}$ : h^\prime \in H~\&~h^\prime \text{ is total}\}$, undefined otherwise;
      \ex\label{semreset} \valsem[g,h]{$X$} $=$ \valsem[g]{$X$}
    \end{xlist}
  \ex \valsem{only}$(\alpha)(\beta)(w)$ $=$ 1\\[0.2cm]
      iff $\forall{p}[~[p(w) = 1 \wedge p \in \alpha] \rightarrow p = \beta]$
\end{exe}
\end{frame}

\begin{frame}\framesecnum
Composition of the sentence \enquote{\emph{Only JOHN left}}:
\begin{exe}
  \ex 
    \begin{xlist}
        \ex \valsem[g]{only} $(g(C))$ $(\lambda{w_s}.~\text{john left in }w)$ $(w)$ $=$ 1 iff
        \ex $\forall{p}[~[p(w) = 1 \wedge p \in g(C)] \rightarrow p = \lambda{w_s}.~\text{john left in } w ]$\\
        ~~~~ if $g(C) \subseteq \{\lambda{w_s}.~x \text{ left in } w : x \in D_e\}$
        \ex $\forall{p}[~[p(w) = 1 \wedge p \in \{\lambda{w_s}.~x \text{ left in } w : x \in D\}] \rightarrow p = \lambda{w_s}.~\text{john left in } w]$
    \end{xlist}
\end{exe}
  
\end{frame}



%%Lucas
\section{Beck's Analysis}
\begin{frame}{Beck's analysis}
\begin{block}{\normalsize Beck's analysis (informal)}
{\footnotesize{Focused phrases make two semantic contributions:}}
    \begin{itemize}
      \footnotesize \item their ordinary semantic value
      \item a set of alternatives of the same type of the ordinary semantic value
    \end{itemize}
{\footnotesize{\emph{wh-} phrases also introduce a set of alternatives, but \emph{they don't have an ordinary semantic value}. Since the ordinary semantic value of a \emph{wh-} phrase is undefined, the role of the question operator $Q$ is to ignore the ordinary semantic value of his sister and to elevate its focus semantic value to an ordinary semantic value.}}
  
{\footnotesize{Things go wrong when the question contains a focus-sensitive operator in the scope of the $Q$ operator, which will try to evaluate the focus on the \emph{wh-} phrase. The problem is that the focus-sensitive operator resets the focus semantic value of its sister to its ordinary semantic value. The $Q$ operator won't have access to the focus semantic value introduced by the \emph{wh-} phrase, and then the whole sentence will be undefined.}}
\end{block}
\end{frame}

\begin{frame}\framesecnum
There is a strong correlation between focus and questions.

\begin{exe}
  \ex Focus
    \begin{xlist}
      \ex \lb{} \hspace{-1.4ex} John left\rb{F}.
      \ex Set of alternatives generated by focus:\\
          {\footnotesize{\{that John left, that Mary left, that Paul left, \ldots \}}}\\
          $\lambda{p_{\tuple{s,t}}}.~\exists{x}[p = \lambda{w_s}.~x \text{ left in } w]$
    \end{xlist}
  \ex Questions
    \begin{xlist}
      \ex Who left?
      \ex {\footnotesize{\{that John left, that Mary left, that Paul left, \ldots \}}}\\
          $\lambda{p_{\tuple{s,t}}}.~\exists{x}[p = \lambda{w_s}.~x \text{ left in } w]$
    \end{xlist}
\end{exe}

The focus semantic value of the sentence \enquote{John_F left} is exactly the same as the ordinary semantic value of the question \enquote{Who left?}.
\end{frame}

\begin{frame}\framesecnum
Beck proposes that a \emph{wh-} phrase does not have an ordinary semantic value, but only a focus semantic value (a set of alternatives of the same type of the ordinary semantic value).
\begin{exe}
  \ex
    \begin{xlist}
      \ex \valsem[g]{who_1} is undefined.
      \ex \valsem[g,h]{who_1} $= h(1)$
    \end{xlist}
  \ex
    \begin{xlist}
      \ex \valsem[g]{who_1 left} is undefined.
      \ex \valsem[g,h]{who_1} left $= \lambda{w_s}.~ h(1) \text{ left in } w$
    \end{xlist}
\end{exe}

Now we can introduce the second focus-sensitive operator besides $\sim$: the $Q$ operator. It is a variable binder that binds variables interpreted by $h$.
\end{frame}

\begin{frame}\framesecnum
$Q$ takes the focus semantic value of a \emph{wh-} word and elevates it to its ordinary semantic value. The \emph{wh-} word now has an ordinary semantic value and can be interpreted normally.

\begin{exe}
    \ex
        \begin{xlist}
            \ex Who left?
            \ex \lb{} $Q_1$ \lb{} who_1 left \rb{} \rb{}
        \end{xlist}
    \ex If $X = [Q_i ~ Y]$, then 
        \begin{xlist}
            \ex \valsem[g]{$X$} $= \lambda{p_{\tuple{s,t}}}.~\exists{x}[p = $\valsem[g,h^{[i \rightarrow x]}]{$Y$}$]$
            \ex and \valsem[g,h]{$X$} $= \lambda{p_{\tuple{s,t}}}.~\exists{x}[p = $\valsem[g,h^{[i \rightarrow x]}]{$Y$}$]$
        \end{xlist}
    \ex \valsem[g]{$Q_i$ [ who_i left ]} $= \lambda{p_{\tuple{s,t}}}.~\exists{x}[p = \lambda{w_s}.~x \text{ left in } w]$
\end{exe}
\end{frame}

\subsection[Deriving Intervention Effects]{Deriving intervention effects in wh- questions}
\begin{frame}{Deriving Intervention effects in wh- questions}
Before deriving intervention effects, it should pointed out that we need a principle that prevents structures without an ordinary semantic value to be interpreted (see \reff{poi}).

\begin{exe}
    \ex\label{poi} \emph{Principle of Interpretability:}\\
        An LF must have an ordinary semantic interpretation
\end{exe}
\end{frame}

\begin{frame}\framesubsecnum{}

Now, let's look at a prototypical example of intervention:

\begin{exe}
    \ex
        \begin{xlist}
            \ex{*}{Only JOHN saw who?}\label{onlyjohnsawwho}
            \ex {\small{\lb{CP} $Q_2$ \lb{IP3} only \lb{IP2} $\sim$ $C$ \lb{IP1} John_{F_{i}} saw who_2 \rb{} \rb{} \rb{} \rb{}}}
        \end{xlist}
\end{exe}

Here, IP1 contains both a focus-marked element and a \emph{wh-} word, the latter lacking an ordinary semantic value. Thus \valsem[g]{IP1} is not defined.

At the level of IP2, the focus-sensitive operator $\sim$ will evaluate all focus features unselectively and reset the focus semantic value of IP1 to its ordinary semantic value. The problem is that IP1 \emph{does not} possess an ordinary semantic value!

Then, \valsem[g]{IP3} is undefined. At the CP level, the highest focus-sensitive operator $Q_2$ cannot evaluate the focus feature on \emph{who_2} since it has been resetted by $\sim$. It cannot elevate the focus semantic value of the whole sentence to its ordinary semantic value, thus \reff{onlyjohnsawwho} is uninterpretable.

\end{frame}

\begin{frame}\framesubsecnum{}
Beck's analysis predicts that the $\sim$ operator acts as an intervener whenever alternative semantics is necessary. She thus proposes the general principle in \reff{genmineff}, where the use of alternative semantics is necessary in $\phi$.

\begin{exe}
  \ex\label{genmineff} \emph{General Minimality Effect}:\\
      The evaluation of alternatives introduced by an XP cannot skip an intervening $\sim$ operator.\\
      * [ $Op_1$ \ldots [ $\sim C$ \lb{\phi} \ldots{} XP_1 \ldots \rb{}] ]
\end{exe}
\end{frame}



\section{Types of movement and intervention}
\begin{frame}{Types of movement and intervention}

\begin{itemize}
\item \cite{Pes00} argues that wh-movement can involve
    \begin{itemize}
    	\item overt or covert phrasal movement at LF\\
	$\Rightarrow$ signaled by superiority effects
	
    \item feature movement \\
	$\Rightarrow$ signaled by absence of superiority effects
    \end{itemize}
\item Beck argues that focus interpretation is the "interpretational strategy that underlies the term feature movement"
    \begin{itemize}
    \item superiority $\Rightarrow$ phrasal movement $\Leftrightarrow$ no intervention 
    \item no superiority $\Rightarrow$ feature movement $\Leftrightarrow$ intervention 
    \end{itemize}
\item The relevant data comes from English intervention effects that only show up in otherwise permissible superiority violations (\cite{Pes00})


\end{itemize}

\end{frame}







\begin{frame}{Phrasal movement: no intervention}

	\begin{exe}
		\ex Who did John introduce \_ to whom? \\
		{\footnotesize{Q_{1,2} [who_{1} [4 [whom_{2} [5 [did [John introduce t4 to t5 ]]]]]]}}
		\ex\label{a} Who did only John introduce \_ to whom? \\
		{\scriptsize{Q_{1,2} [who_{1} [4 [whom_{2} [5 [did [_{X} only_{C} [$\sim$C [John_{F3} introduce t4 to t5 ]]]]]]]]}}\label{intr}
	\end{exe}
	
The in-situ wh-phrase moves covertly, as shown by superiority 
    \begin{exe}
    \ex[*]{Who did John introduce who to \_ ?}
    \end{exe}
This means that it is interpreted outside the scope of $\sim$ in (\ref{intr}), and crucially, $X$ is defined\\ $\Rightarrow$ \textbf{No intervention}
\end{frame}


\begin{frame}{Feature movement: intervention}

	\begin{exe}
		\ex Which boy did Mary introduce which girl to \_ ?\\
		{\footnotesize{Q_{1,2} [ [which boy]_{1} [4 [did [Mary introduce [which girl]_{2} to t4 ]]]]]]}}
		\ex[??]{Which boy did only Mary introduce which girl to \_ ?}
		{\scriptsize{Q_{1,2} [ [which boy]_{1} [4 [did [_{X} only_{C} [$\sim$C [Mary_{F3} introduce [which girl]_{2} to t4 ]]]]]]}}
	\end{exe}

The in-situ wh-phrase does not move overtly or covertly at LF, as shown by the lack of superiority effects. It is therefore interpreted inside the scope of $\sim$, and crucially, \textit{X} is undefined\\ $\Rightarrow$ \textbf{Intervention}
\end{frame}

\begin{frame}{Focus interpretation = feature movement?}

Beck: "[focus interpretation] is an interpretation of the notion of feature movement as used by Pesetsky", "a semantic reconstruction of the use that feature movement is put to by Pesetsky"

\begin{itemize}
\item If the focus intervention account is right, focus intervention is not the correlate but the consequence of non-phrasal movement: the problem is not that a feature moves, but that the wh-phrase doesn't move
\item When only the wh-feature moves, the wh-expression is evaluated by $\sim$ and undefinedness follows
\item When the wh-expression itself moves, semantic composition is not disrupted
\end{itemize}


\end{frame}


%%karo
\begin{frame}{Predictions}
\begin{itemize}
\item Japanese, Korean, German...
    \begin{itemize}
    \item no superiority $\Rightarrow$ no phrasal movement available $\Rightarrow$ feature movement $\Rightarrow$ generalised intervention effects
    \end{itemize}
\item English, ...
    \begin{itemize}
    \item superiority $\Rightarrow$ phrasal movement available $\Rightarrow$ limited intervention effects
    \end{itemize}
\item Intervention when $\sim$ is present 
    \begin{itemize}
    \item According to \cite{Truc95}, the $\sim$ has phonological consequences
    \item Then, we should observe phonological effects in correlation with intervention effects
    \end{itemize}
\end{itemize}
\end{frame}




\section{RM vs. FI}
\begin{frame}{(Featural) Relativized Minimality}
\cite{Rizzi2013} (based on \cite{Rizzi1990})
\vspace{1em}

[X...Z...Y] 

\begin{exe}

\ex A local relation (e.g., movement) cannot hold between X and Y if
    \begin{xlist}
    \ex Z intervenes 
    \ex Z fully matches the specification of X in the relevant morphosyntactic features
    \end{xlist}
\end{exe}

\begin{exe}
\ex From good (a) to bad (c):
    \begin{xlist}
    \ex disjunction: X_{A}...Z_{B}...Y_{A}
    \ex inclusion: X_{A,B}...Z_{A}...Y_{A,B}
    \ex identity: X_{A}...Z_{A}...Y_{A}
    \end{xlist}
\end{exe}
\end{frame}


\begin{frame}{D-linking, RM and focus interpretation}

D-linked wh-phrases do not show superiority effects (\ref{rm1}-\ref{rm2}). However, \textit{only} intervenes in such questions (\ref{rm3}): 


\begin{exe}
\ex[\footnotesize *]{\footnotesize{\gll What do you think who bought t ?\\
[+Q] {} {} {} [+Q] {} [+Q] \\
}}\label{rm1}
\ex[\footnotesize ?]{\footnotesize{\gll Which problem do you wonder how to solve t ? \\
[+Q, +N] {} {} {} [+Q] {} {} {[+Q, +N]} \\
}}\label{rm2}

\ex[\footnotesize *]{\footnotesize{\gll Which book did who only read t ? \\
[+Q, +N] {}  [+Q] [+Q] {} {[+Q, +N]} \\
}}\label{rm3}
\end{exe}

\begin{itemize}
\item In RM terms, the intervenor \textit{only} should either be featurally identical to the wh-phrase trace, or featurally richer than it
\item Under a focus intervention account, the ungrammaticality should be due to a wh-phrase being evaluated by $\sim$, leading to an undefined ordinary semantic value that is inherited by the whole question
\end{itemize}
\end{frame}



\begin{frame}{D-linking, RM and focus interpretation (II)}

\begin{exe}
\exr{rm3}[\footnotesize *]{\footnotesize{\gll Which book did who only read t ? \\
[+Q, +N] {}  [+Q] [+Q] {} {[+Q, +N]} \\
}}
\end{exe}

RM:
\begin{itemize}
\item Let's assume the RM-featural composition of \textit{only} is [+Q]
\item The trace has more features than the intervenor
\item No RM intervention expected
\end{itemize}

Focus intervention:
\begin{itemize}
\item No superiority $\Rightarrow$ no phrasal movement $\Rightarrow$ feature movement $\Rightarrow$ intervention
\item If the in situ \textit{who} moves featurally to Q, it moves from \textbf{above} only, and no intervention is expected
\item Intervention is only expected if \textit{which book}, although fronted in surface syntax, moves only featurally at LF
\end{itemize}

\end{frame}



\begin{frame}{Comparing feature movement examples}
Coming back to the feature movement/intervention example: as the structurally lower \textit{which boy} can be fronted without superiority effects, the in situ \textit{which girl} must move featurally

\begin{exe}
		\ex Which boy did Mary introduce which girl to \_ ?\\
		{\footnotesize{Q_{1,2} [ [which boy]_{1} [4 [did [Mary introduce [which girl]_{2} to t4 ]]]]]]}}
		\ex[??]{Which boy did only Mary introduce which girl to \_ ?}
		{\scriptsize{Q_{1,2} [ [which boy]_{1} [4 [did [_{X} only_{C} [$\sim$C [Mary_{F3} introduce [which girl]_{2} to t4 ]]]]]]}}\label{which}
	\end{exe}

\begin{itemize} 
\item The position of \textit{only} in (\ref{which}) is different from (\ref{rm3}): here, \textit{only} c-commands both the in situ wh-phrase and the trace, while in (\ref{rm3}), \textit{only} only c-commands the trace.

\item As (\ref{rm3}) shows an intervention effect that on the FI account can only be explained by saying that the overtly fronted wh-phrase is in situ at LF, it doesn't seem possible to say for (\ref{which}) whether the intervention effect is due to \textit{which girl} or \textit{which boy}
\end{itemize}
\end{frame}


\begin{frame}{Comparing feature movement examples (II)}

Direct vs indirect object asymmetry: D-linked \textit{which}-phrases \textbf{can} sometimes do phrasal movement (\cite{Pes00}) 

\begin{exe}
		\ex[??]{Which boy did only Mary introduce which girl to \_ ?}\label{indirect}
		\ex\label{direct} Which girl did only Mary introduce \_ to which boy ?
	\end{exe}

\begin{itemize} 
\item According to \cite{Pes00}, (\ref{direct}) is acceptable: Beck's informants perceive no difference between (\ref{indirect}) and (\ref{direct})
\item Phrasal movement at LF of both objects in (\ref{direct})
\item No phrasal movement at LF of at least one of the objects in (\ref{indirect})
    \begin{itemize}
    \item Two dialects of English? One where no which-phrase moves phrasally, and another where it can sometimes move? Constraints?
    \end{itemize}

\end{itemize}
\end{frame}



\section[Conclusion]{Conclusion and Open Issues}
\begin{frame}{Conclusion and Open Issues}
\begin{itemize}
\item \emph{wh-} words and phrases that bear a focus feature both have a focus semantic value. Only the latter has an ordinary semantic value, while the former's ordinary semantic value is undefined.
\item The two focus-sensitive operators in Beck's framework are the squiggle $\sim$, and the question operator $Q$.
\item Whenever there is a configuration where the first c-commanding element of a \emph{wh-} word is $\sim$, and not the question operator $Q$ (as in \reff{intervconfig}), an intervention effect will rise.
\end{itemize}

\begin{exe}
  \ex\label{intervconfig} \emph{Intervention Configuration}:\\
      * [ $Op_1$ \ldots [ $\sim C$ \lb{\phi} \ldots{} \emph{wh-}_{{1}} \ldots \rb{}] ]
\end{exe}
\end{frame}

\begin{frame}\framesecnum{}
This universal principle is subject to variation crosslinguistically, across languages, the set of interveners is not the same.

The interveners that are supposed to be universal are \emph{only}, \emph{even} and \emph{also}.
\end{frame}

\begin{frame}\framesecnum{}
\begin{center}
\LARGE{But...}
\end{center}
\end{frame}


\begin{frame}\framesecnum{}
  FRENCH!!
  
 {\small{ 
 \textbf{Dialogue 1:}\\
A: Jean devait voir plusieurs personnes aujourd'hui, qu'est-ce qu'il en est?\\
B: Je crois qu'il n'a pas eu le temps de voir tout le monde. Je crois qu'il a vu une personne et après il était débordé.\\
A: Dis moi, au final, Jean a vu seulement qui?


\textbf{Dialogue 2:}\\
A: C'était compliqué d'organiser la fête chez moi! Je ne pouvais pas inviter tout le monde, et j'ai eu du mal à choisir les gens que je n'allais pas inviter. Mais maintenant c'est réglé!\\
B: Alors au final, t'as pas invité qui?


\textbf{Dialogue 3:}\\
A: Pour la fête qu'il organise chez lui, Jean est complètement fou! Il a invité beaucoup de monde, il a même invité des gens très bizarres...\\
B: Ah, il a même invité qui par exemple?}}
\end{frame}

\begin{frame}\framesecnum
\begin{center}
\begin{tabular}{cccc}
 \cline{2-4}
    & \textsc{only} & \textsc{negation} & \textsc{even}\\
 \hline
 \checkmark & 4 & 5 & 4 \\
 \hline
 ? & 1 & 1 & 1 \\
 \hline
 * & 1 & 0 & 2 \\
 \hline
\end{tabular}
\end{center}
\end{frame}

\begin{frame}{References}
  \begin{scriptsize}
   \bibliography{presbeckbib}
  \end{scriptsize}
  \bibliographystyle{apalike}
  \nocite{*}
\end{frame}

\end{document}
